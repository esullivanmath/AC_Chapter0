\begin{exercises} 

\item Suppose that $h(t) = A \cdot r^t$.  If $h(3)=4$ and $h(5)=40$,
    \ba
        \item find $r$.
        \item find $A$.
        \item Does this function model exponential growth or decay? How can you tell?
    \ea
\begin{exerciseSolution}
\end{exerciseSolution}


\item The half-life of $Br^{77}$ is 57 hours.
    \ba
        \item If the initial amount is $150$ grams, find the amount remaining after 171
            hours.
        \item Write an equation to predict the amount remaining after $t$ hours.
        \item Estimate within one hour how long it will take the amount to decrease to 10
            grams.
    \ea
\begin{exerciseSolution}
\end{exerciseSolution}


\item Consider the data in Table \ref{tab:0.2.exercise3}
    \ba
        \item Which (if any) of the functions could be linear? Explain how you know that
            these functions are linear, and find formulas for these functions.
        \item Which (if any) of the functions could be exponential? Explain how you know
            that these functions are linear, and find formulas for these functions.
    \ea
    \begin{table}[h!]
        \centering
        \begin{tabular}{|c|c|c|c|}
            \hline
            $x$ & $f(x)$ & $g(x)$ & $h(x)$ \\ \hline
           $-2$&$12$&$16$&$37$\\
           $-1$&$17$&$24$&$34$\\
           $0 $&$20$&$36$&$31$\\
           $1 $&$21$&$54$&$28$\\
           $2 $&$18$&$81$&$25$\\ \hline
        \end{tabular}
        \caption{Data tables for $f(x)$, $g(x)$, and $h(x)$}
        \label{tab:0.2.exercise3}
    \end{table}
\begin{exerciseSolution}
\end{exerciseSolution}

\end{exercises}
\afterexercises
