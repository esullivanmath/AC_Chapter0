\begin{exercises} 

\item (modified from NCTM Illuminations) The table below displays data that relate the number of oil changes per year and the
    cost of engine repairs.  To predict the cost of repairs from the number of oil
    changes, use the number of oil changes as the $x$ variable and the engine repair cost
    as the $y$ variable.  
    \begin{center}
        \begin{tabular}[h!]{|c|c|}
            \hline
            Oil Changes Per Year & Cost of Repairs (\$) \\ \hline \hline
            3 & 300 \\ 
            5 & 300 \\ 
            2 & 500 \\
            3 & 400 \\ 
            1 & 700 \\
            4 & 400 \\
            6 & 100 \\
            4 & 250 \\ 
            3 & 450 \\
            2 & 650 \\
            0 & 600 \\ 
            10 & 0 \\
            7 & 150 \\ \hline
        \end{tabular}
    \end{center}

    \ba
    \item Using graph paper make a plot of the data on appropriate axes.
    \item Do the data appear linear?  Why or why not?
    \item Pick two representative points from the data and use them to write the
        equation of a line that {\it fits} the data.  Plot your line on top of your data
        and discuss how well your line fits the
        data.   (This may take a few attempts.)
    \item Despite how well your data fit a linear model, it is not entirely sensible to
        use a linear model for this data.  Why?
    \ea
    
\begin{exerciseSolution}
\end{exerciseSolution}


\end{exercises}
\afterexercises
