\begin{activity}\label{A:0.4.1}
    Use the definition of a logarithm along with the properties of logarithms to answer
    the following.
    \ba
\item Write the exponential expression $8^{1/3} = 2$ as a logarithmic expression.
\item Write the logarithmic expression $\log_2 \frac{1}{32} = -5$ as an exponential
    expression.
\item What value of $x$ solves the equation $\log_2 x = 3$?
\item What value of $x$ solves the equation $\log_2 4 = x$?
\item Use the laws of logarithms to rewrite the expression $\log \left( x^3 y^5 \right)$
    in a form with no logarithms of products, quotients, or powers.
\item Use the laws of logarithms to rewrite the expression $\log \left( \frac{x^{15}
    y^{20}}{z^4} \right)$
    in a form with no logarithms of products, quotients, or powers.
\item Rewrite the expression $\ln(8) + 5 \ln(x) + 15 \ln(x^2+8)$ as a single logarithm.
    \ea


\end{activity}
\begin{smallhint}
    Use the properties of logarithms.
\end{smallhint}
\begin{bighint}
    Use the properties of logarithms.
\end{bighint}
\begin{activitySolution}
    \ba
\item $8^{1/3} = 2$ is equivalent to $\log_8 2 = \frac{1}{3}$.
\item $\log_2 \frac{1}{32} = -5$ is equivalent to $2^{-5} = \frac{1}{32}$.
\item $\log_2 x = 3$ is equivalent to $x = 2^3 = 8$.
\item $\log_2 4 = x$ is equivalent to $2^x = 4$ and we see that $x=2$ since $2^2 = 4$.
\item Using the product and power properties
    \[ \log\left( x^3 y^5 \right) = 3\log(x) + 5\log(y). \]
\item Using the product, power, and quotient properties
    \[ \log\left( \frac{x^{15} y^{20}}{z^4} \right) = 15\log(x) + 20\log(y) - 4\log(z). \]
\item Using the power and product properties
    \[ \ln(8) + 5 \ln(x) + 15\ln(x^2+8) = \ln\left( 8 x^5 \left( x^2 + 8 \right)^{15}
        \right) \]
    \ea
\end{activitySolution}

\aftera
