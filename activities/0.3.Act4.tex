\begin{activity}\label{A:0.3.4}
    \ba 
        \item Find the inverse of each of the following functions by interchanging the $x$ and $y$
    and solving for $y$.  Be sure to state the domain for each of your answers.
    \[ y = \sqrt{x-1}, \quad y = -\frac{1}{3} x + 1, \quad y = \frac{x+4}{2x-5} \]
\item Verify that the functions $f(x) = 3x-2$ and $g(x) = \frac{x}{3} +
    \frac{2}{3}$ are inverses of each other by computing $f(g(x))$ and $g(f(x))$.
    \ea
\end{activity}
\begin{smallhint}
    \ba
        \item Switch the $x$ and $y$ and solve for $y$.  It may also be helpful to draw a
            picture.
        \item What should happen to $f(g(x))$ and $g(f(x))$ if $f$ and $g$ are inverses?
    \ea
\end{smallhint}
\begin{bighint}
    \ba
        \item Switch the $x$ and $y$ and solve for $y$.  It may also be helpful to draw a
            picture.
        \item What should happen to $f(g(x))$ and $g(f(x))$ if $f$ and $g$ are inverses?
    \ea
\end{bighint}
\begin{activitySolution}
    \ba
\item 
    \begin{itemize}
        \item For $y=\sqrt{x-1}$: First switch the $x$ and $y$ to get $x = \sqrt{y-1}$.
            Then if we square both sides and add 1 we get $y = x^2 + 1$.  Notice
            graphically that the inverse function only makes sense for $x \ge 0$.
        \item For $y = -\frac{1}{3}x + 1$: First switch the $x$ and $y$ to get $x =
            -\frac{1}{3}y + 1$.  Then if we subtract $1$ and multiply by $-3$ we get $y =
            -3(x-1) = -3x+3$.  This linear function makes sense for all $x$.
        \item For $y=\frac{x+4}{2x-5}$: First switch the $x$ and $y$ to get $x =
            \frac{y+4}{2y-5}$.  Multiplying by the denominator gives $x(2y-5) = y+4$, and
            then if we distribute the $x$ we get $2xy - 5x = y+4$.  Now we should gather
            all of the $y$ terms on the same side to get $2xy - y = 5x + 4$.  Factoring
            the $y$ out from the left and dividing gives $y = \frac{5x+4}{2x-1}$. This
            function makes sense for all $x$ not equal to $1/2$.
    \end{itemize}
        \item If we compose the two functions then
            \[ f(g(x)) = 3\left( \frac{x}{3} + \frac{2}{3} \right) -2  = x + 2 -2 = x. \]
            \[ g(f(x)) = \frac{1}{3} \left( 3x-2 \right) + \frac{2}{3} = x -
                \frac{2}{3} + \frac{2}{3} = x. \]
            Since both $f(g(x)) = x$ and $g(f(x)) = x$ we know that $f$ and $g$ are
            inverses of each other.
    \ea
\end{activitySolution}

\aftera
