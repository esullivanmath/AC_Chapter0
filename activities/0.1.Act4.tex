\begin{activity}\label{A:0.1.4}
Write the equation of the line with the given information.
\ba
\item Write the equation of a line parallel to the line $y=\frac{1}{2}x+3$ passing through
    the point $(3,4)$.
\item Write the equation of a line perpendicular to the line $y=\frac{1}{2}x + 3$ passing
    through the point $(3,4)$.
\item Write the equation of a line with $y$-intercept $(0,-3)$ that is perpendicular to
    the line $y=-3x-1$.
\ea
\end{activity}
\begin{smallhint}
    \ba
        \item What is the slope of the line that you're given and which form of a linear
            function would be most convenient?
        \item What is the slope of the line that you're given and which form of a linear
            function would be most convenient?
        \item You are given the $y$-intercept.  Why is that a special case?
    \ea
\end{smallhint}
\begin{bighint}
    \ba
        \item The slope should be the same since the lines are parallel.
        \item The slope should be the opposite reciprocal since the lines are
            perpendicular.
        \item The slope should be the opposite reciprocal since the lines are
            perpendicular.
    \ea
\end{bighint}
\begin{activitySolution}
    \ba
        \item The slope is $m = \frac{1}{2}$ and you have a point so use the point-slope
            form of the line to get
            \[ y - 4 = \frac{1}{2} \left( x-3 \right). \]
            Solving for $y$ we get 
            \[ y = \frac{1}{2} \left( x-3 \right) + 4. \]
        \item The slope is $m = -2$ and we have a point so use the point-slope form of the
            line to get
            \[ y - 4 = -2 (x-3). \]
            Solving for $y$ we get 
            \[ y = -2(x-3) + 4. \]
        \item The slope is $m = \frac{1}{3}$ and since we have the $y$-intercept we know
            that the slope-intercept for the line is the proper choice.  Hence,
            \[ y = \frac{1}{3} x - 3. \]
    \ea
\end{activitySolution}



\aftera
