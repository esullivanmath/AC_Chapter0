\begin{activity}\label{A:0.6.5}
	For each of the following functions, determine (1) whether the function has a horizontal asymptote, and (2) whether the function crosses its horizontal asymptote.
\ba
		\item $f(x)=\displaystyle{\frac {x+3}{5x-2}}$
		\item $g(x)=\displaystyle{\frac {x^{2}+2x-1}{x-1}}$
		\item $h(x)=\displaystyle{\frac {x+1}{x^{2}+2x-1}}$
\ea
\end{activity}
\begin{smallhint}
    A horizontal asymptote occurs where the function approaches a value  as $x\to\infty$.
    It is possible for a function to cross a horizontal asymptote.
\end{smallhint}
\begin{bighint}
    A horizontal asymptote occurs where the function approaches a value  as $x\to\infty$.
    It is possible for a function to cross a horizontal asymptote.
\end{bighint}
\begin{activitySolution}
   \ba
        \item For $f(x) = \frac{x+3}{5x-2}$ the degrees of the two polynomials are both $1$
            so the numerator and denominator grow at the same rate.  The horizontal
            asymptote will be the ratio of the coefficients: $f(x) \to
            \frac{1}{5}$ as $x \to \infty$.  The graph does not cross the horizontal
            asymptote.
        \item In this case where $g(x) = \frac{x^2+2x-1}{x-1}$ the degree of the numerator
            is larger than the degree of the denominator so $g(x) \to \infty$ as
            $x\to\infty$ and there is no horizontal asymptote.
        \item In this case where $h(x) = \frac{x+1}{x^2+2x-1}$ the degree of the
            denominator is larger than the degree of the numerator so $h(x) \to 0$ as $x
            \to \infty$.  Therefore, there is a horizontal asymptote at $y=0$.  Observe
            that $h(-1) = 0$ so the function's graph does indeed cross the horizontal
            asymptote.
   \ea
\end{activitySolution}

\aftera
